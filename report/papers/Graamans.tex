\section{Crop energy balance}
We extrapolated from paper Graamans the energy balance of the crop.\\
At the base of this treatment there is an important assumption introduced by Penman \& Monteith (1965): the three dimentional crop canopy is reduced to a one-dimentional \emph{big leaf} where net radiation is absorbed, heat is exchanged and water vapour escapes due to evapotranspiration.\\
We are allowed to make this assumption since in our cell crop is homogeneus, level, continuous and extensive enough.\\
At equilibrium, the energy arriving at the \emph{big leaf} equals the amount leaving it:
\begin{equation}	\label{crop energy balance}
	R_{net} - H - \lambda E = 0
\end{equation}
Where:
\begin{itemize}
	\item $R_{net}$ [$Wm^{-2}$] is the net radiation absorbed by the crop;
	\item $H$ [$Wm^{-2}$] is the sensible heat flux;
	\item $\lambda E$ [$Wm^{-2}$] is the latent heat flux, thus the heat related with the evapotranspiration of the lettuce.
\end{itemize}
Clearly $R_{net}$ is a data, since we can determine it after the light intensity is set, in the paper the following relation is presented:
\begin{equation}	\label{R_net}
	R_{net} = (1-\rho_r)I_{lighting}CAC
\end{equation}
where the coefficients are known:
\begin{itemize}
	\item $\rho_r \approx{0.05 - 0.08}$ is the reflection coefficient;
	\item $CAC \approx{0.9}$ is the ratio of projected leaf area to cultivation area, which is almost constant in lettuce throughout its development.
\end{itemize}
and $I_{lighting}$ [$Wm^{-2}$] is a control variable.\\
On the other hand, for the heat fluxes, Gramaans presents the following relations:
\begin{equation}	\label{heat fluxes}
	\begin{cases}
		H = LAI \rho_a C_p \frac{T_c - T_a}{r_a}\\
		\lambda E = LAI \lambda \frac{\gamma_{H_2O,sur} - \gamma_{H_2O,air}}{r_s + r_a}\\
	\end{cases}
\end{equation}
Where:
\begin{itemize}
	\item $r_a$ [$sm^{-1}$] is the aereodynamics resistence to heat;
	\item $r_s$ [$sm^{-1}$] is the stomatal resistence of vapour flow through the transpiring crop;
	\item $\lambda$ [$Jg^{-1}$]is the latent heat of the evaporation of water and it is unknown.
\end{itemize}
Remark that we already computed the values of $LAI$ (at \eqref{LAI}), $\gamma_{H_2O,sur}$ and $\gamma_{H_2O,air}$ (at \eqref{air mass fractions}); moreover $\rho_a$ and $C_p$ are given data already used in \eqref{Navier-Stokes} and \eqref{air energy balance}.\\
Notice also that we made the assumption \eqref{canopy temperature}, thus the numerator of the fraction inside \eqref{heat fluxes} in the $H$ equation is 1.\\
About the difference $\gamma_{H_2O,sur} - \gamma_{H_2O,air}$ we can state that we are able to determine it since we already know from \eqref{air mass fractions} each $H_2O$ mass fraction. Anyway, Graamans presents a relation between the two, holding if assuming that the transpiring surface is saturated at his temperature, which may save us some previous computations if adopted:
\begin{equation}	\label{saturation hypotesis}
	 \gamma_{H_2O,sur} \approx{ \gamma_{H_2O,air} + \frac{\rho_a C_p}{\lambda} \epsilon (T_c-T_a) }
\end{equation}
In the end, the stomatal resistance is known and it depends on the photosyntetic photon flux density $PPFD$ [$\mu mol m^{-2}s^{-1}$], which is a control variable.
On the other hand, the aereodynamics resistance can be determined using a relation introduced by Fuchs (1993).\\
Explicitly:
\begin{equation}	\label{graamans resistences}
	\begin{cases}
		r_s = 60 \frac{1500 + PPFD}{200 + PPFD}\\
		r_a = 350 (\frac{l}{u_\infty})^{0.5} LAI^{-1}
	\end{cases}
\end{equation}
with $l$ being the mean leaf diameter and $u_\infty$ the uninhibited air speed, both known data.
	
\section{Evapotranspiration}
After having determined the flux of latent heat, we are able to compute in the end the value of our interest: $ET$ [$gm^{-2}s^{-1}$], which represents the amount of water vapour released by the crop due to the evapotranspiration process.